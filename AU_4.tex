\documentclass[xcolor=dvipsnames]{beamer}
%\documentclass{beamer}

%\documentclass[handout]{beamer}

%\usepackage[table]{xcolor}
\mode<presentation> {
  \usetheme{Boadilla}
%  \usetheme{Pittsburgh}
%\usefonttheme[2]{sans}
\renewcommand{\familydefault}{cmss}
%\usepackage{lmodern}
%\usepackage[T1]{fontenc}
%\usepackage{palatino}
%\usepackage{cmbright}
  \setbeamercovered{transparent}
\useinnertheme{rectangles}
}
\usepackage{soul}
\setbeamercolor{normal text}{fg=black}
\setbeamercolor{structure}{fg= black}
\beamertemplatesolidbackgroundcolor{white}  \setbeamercolor{alerted
text}{fg=red}
\usepackage{amsmath}
\usepackage[english]{babel}
\usepackage[latin1]{inputenc}
\usepackage{times}
\usepackage[T1]{fontenc}
\usepackage{colortbl}
\usepackage{color}
\definecolor{gold}{rgb}{0.85,0.66,0}
\usepackage{cancel}
\usepackage{comment}
\usepackage{enumerate}
\usepackage{multirow}
\usepackage{fancybox}
\usepackage{framed}

% === check mark
\usepackage{pifont}
\newcommand{\cmark}{\ding{51}}
\newcommand{\xmark}{\ding{55}}

% === tikz for pictures ===
\usepackage{tikz}
\usepackage[latin1]{inputenc}
\usetikzlibrary{shapes,arrows,trees,fit,positioning}

% ==== dotted lines in tables ===
\usepackage{arydshln}
\usepackage[normalem]{ulem}

% === dcolumn package ===
\usepackage{dcolumn}
\newcolumntype{.}{D{.}{.}{-1}}
\newcolumntype{d}[1]{D{.}{.}{#1}}

% === new commands ===
\newcommand\ud{\mathrm{d}}
\newcommand\dist{\buildrel\rm d\over\sim}
\newcommand\ind{\stackrel{\rm indep.}{\sim}}
\newcommand\iid{\stackrel{\rm i.i.d.}{\sim}}
\newcommand\logit{{\rm logit}}
\renewcommand\r{\right}
\renewcommand\l{\left}
\newcommand\Var{{\rm Var}}
\newcommand\var{{\rm var}}
\newcommand\Cov{{\rm Cov}}
\newcommand\bone{\mathbf{1}}
\newcommand\E{\mathbb{E}}
\newcommand\wX{\widetilde{X}}
\newcommand\wT{\widetilde{T}}
\newcommand\independent{\protect\mathpalette{\protect\independenT}{\perp}}
\def\independenT#1#2{\mathrel{\rlap{$#1#2$}\mkern2mu{#1#2}}}
\usetikzlibrary{shapes,backgrounds}
\tikzstyle{cblue}=[circle, draw, thin,fill=cyan!20, scale=0.8]
\tikzstyle{qgre}=[rectangle, draw, thin,fill=green!20, scale=0.8]
\tikzstyle{rpath}=[ultra thick, red, opacity=0.4]
\tikzstyle{legend_isps}=[rectangle, rounded corners, thin,
                       fill=gray!20, text=blue, draw]
\usetikzlibrary{decorations.pathreplacing}
\tikzset{text/.default=}
%\tikzset{text/.align=0}
\tikzstyle{every picture}+=[remember picture]
\tikzstyle{na} = [baseline=-.5ex]

\usetikzlibrary{shapes}
\usetikzlibrary{positioning}
\usetikzlibrary{automata}
\usepackage{amsmath,amssymb,amsfonts,amsthm}
\setbeamercovered{invisible} %% <--- I ADDED THIS
\newcommand{\red}{\textcolor{red}}
\newcommand{\blue}{\textcolor{blue}}
\newcommand{\purple}{\textcolor{purple}}
\newcommand{\brown}{\textcolor{brown}}
\newcommand{\cyan}{\textcolor{cyan}}
\newcommand{\real}{\ensuremath{\mathbb{R}}}
\newcommand{\y}{\ensuremath{\mathbf{y}}}
\newcommand{\black}{\color{black}}
\newcommand{\btheta}{\boldsymbol{\theta}}
\newcommand{\green}{\color{green}}
\newcommand{\word}[1]{\green{\textit{#1}\ }\black}
\newcommand{\lb}{\linebreak}
\newcommand\K{{\bm K}}

 \newenvironment{changemargin}[3]{%
 \begin{list}{}{%
 \setlength{\topsep}{0pt}%
 \setlength{\leftmargin}{#1}%
 \setlength{\rightmargin}{#2}%
 \setlength{\topmargin}{#3}%
 \setlength{\listparindent}{\parindent}%
 \setlength{\itemindent}{\parindent}%
 \setlength{\parsep}{\parskip}%
 }%
 \item[]}{\end{list}}


\newcommand\bX{\mathbf{X}}
\newcommand\bB{\mathbf{B}}
\newcommand\bD{\mathbf{D}}
\newcommand\bM{\mathbf{M}}
\newcommand\bH{\mathbf{H}}
\newcommand\bI{\mathbf{I}}
\newcommand\bG{\mathbf{G}}
\newcommand\bR{\mathbf{R}}
\newcommand\bS{\mathbf{S}}
\newcommand\bV{\mathbf{V}}
\newcommand\bW{\mathbf{W}}
\newcommand{\argmin}{\operatornamewithlimits{argmin}}
\newcommand{\argmax}{\operatornamewithlimits{argmax}}
\newcommand{\indep}{\mbox{$\perp\!\!\!\perp$}}
\def\independenT#1#2{\mathrel{\rlap{$#1#2$}\mkern2mu{#1#2}}}
\DeclareMathOperator{\sgn}{sgn}
\newcommand\spacingset[1]{\renewcommand{\baselinestretch}%
{#1}\small\normalsize}
\newcommand\ex{\colorbox{princetonorange}{\color{princetonblack}\textsc{Example}} }
\definecolor{princetonorange}{RGB}{245, 128, 37}
\definecolor{princetonblack}{RGB}{0,0,0}

% == theorems
\setbeamertemplate{theorems}[numbered]
\newcounter{asm}
\setcounter{asm}{0}
\newtheorem{assumption}[asm]{Assumption}
\newtheorem{prop}{Proposition}

%\usepackage[latin1]{inputenc}
\title[Text as Data] % (optional, nur bei langen Titeln nötig)
{Text as Data}

\author{Justin Grimmer}
\institute[Stanford University]{Professor\\Department of Political Science \\  Stanford University}
\vspace{0.3in}


\date{May 24th, 2019}%[Big Data Workshop]
%\date{\today}



\begin{document}
\begin{frame}
\titlepage
\end{frame}



\begin{frame}
\frametitle{Discovery and Measurement}

What is the research process? (Grimmer, Roberts, and Stewart 2017)

\begin{itemize}
  \item[1)] \alert{Discovery}: a hypothesis or view of the world
  \item[2)] \alert{Measurement} according to some organization
  \item[3)] \alert{Causal Inference}: effect of some intervention
\end{itemize}

Text as data methods assist at each stage of research process

\end{frame}



\begin{frame}

\huge

Causal Inference

\end{frame}






\begin{frame}
\frametitle{A Causal Inference Refresher}

\pause 

\begin{itemize}
\invisible<1>{\item[-] Suppose we have $N$ units, $i = 1, \hdots, N$.\\} \pause 
\invisible<1-2>{\item[-] Each unit receives treatment $\boldsymbol{T}_{i}$ (potentially vector valued)} \pause 
\invisible<1-3>{\item[-] Suppose we have (potentially vector valued) response $\boldsymbol{Y}_{i}(\boldsymbol{T}_{i})$} \pause 
\begin{itemize}
\invisible<1-4>{\item[-] Unit $i$'s response to treatment $\boldsymbol{T}_{i}$ } \pause 
\invisible<1-5>{\item[-] Function: maps from treatments to responses} \pause 
\end{itemize}
\invisible<1-6>{\item[-] Fundamental problem of causal inference observe only one $\boldsymbol{Y}_{i}(\boldsymbol{T}_{i})$ } \pause 
\invisible<1-7>{\item[-] Quantity of interest:} \pause 
\begin{itemize}
\invisible<1-8>{\item[-] Simplest (yet very powerful) case$\leadsto$ $T_{i} \in \{0, 1\}$, $Y_{i}(T_{i}) \in \{0, 1\}$} \pause 
\begin{eqnarray}
\invisible<1-9>{\text{ATE} & = & E[Y(1) - Y(0)]  \nonumber } \pause 
\end{eqnarray}
\invisible<1-10>{\item[-] Estimate with: } \pause 
\begin{eqnarray}
\only<1-12>{\invisible<1-11>{\widehat{\text{ATE}} & = & E[Y(1)| T= 1] - E[Y(0)| T = 0 ] \nonumber } }
\only<13->{\widehat{\text{ATE}} & = & \frac{\sum_{i=1}^{N} I(T_{i} = 1) Y_{i}}{ \sum_{i=1}^{N} I(T_{i} = 1) } - \frac{\sum_{i=1}^{N} I(T_{i} = 0) Y_{i}}{ \sum_{i=1}^{N} I(T_{i} = 0) }} \nonumber 
\end{eqnarray}
\pause 
\end{itemize}
\end{itemize}

\pause 
\invisible<1-13>{Question: how do we \alert{accurately} estimate quantities like ATE?} 

\end{frame}

\begin{frame}
\frametitle{Our Plan for the Day}

\begin{itemize}
  \item[-] Experimental design
  \item[-] Conditional average treatment effects
  \item[-] Methods for estimating heterogeneous treatment effects
\end{itemize} 
\end{frame}



\begin{frame}
\frametitle{\alert{An Example} Experiment } 


\only<1-2>{\invisible<1>{Rep. Harold ``Hal" Rogers (KY-05) announced today that Kentucky is slated to
receive \$962,500 to protect critical infrastructure- power plants, chemical
facilities, stadiums, and other high-risk assets, through the U.S. Department of
Homeland Security's buffer zone protection program} }

\only<3>{A federal grant will help keep the Brainerd Lakes Airport operating in winter
weather. Today, Congressman Jim Oberstar announced that the Federal Aviation
Administration (FAA) will award \$528,873 to the Brainerd airport. The funding
will be used to purchase new snow removal and deicing equipment.}


\only<4>{Congresswoman Darlene Hooley (OR-5) and Congressmen Earl Blumenauer (OR-3),
David Wu (OR-1) and Greg Walden (OR-2) joined together today in announcing
\$375,000 in federal funding for the Oregon Partnership to combat methamphetamine
abuse in Oregon.\\}


\only<5>{What information in credit claiming messages affect evaluations?}

\pause \pause \pause \pause


\end{frame}



\begin{frame}
\frametitle{Rewarding Actions and Type of Expenditure, Not Money}

Experiment: vary the \alert{recipient} of money and the \alert{action} reported in credit claiming statement (and many other features)

\invisible<1>{Treatments: \alert<2>{type}}\invisible<1-2>{, \alert<3>{stage}}\invisible<1-3>{, \alert<4>{money}}\invisible<1-4>{, \alert<5>{collaboration}}\invisible<1-5>{, \alert<6>{partisanship}} 

%\invisible<1>{Treatments: \alert<2>{type}}\invisible<1-2>{, \alert<3>{stage}} \invisible<1-3>{money, collaboration, and partisanship} 

\only<1>{\vspace{1.65in}}
\only<7>{\alert{Control Condition}:}


\only<2>{\begin{itemize}
\item[1)] Planned Parenthood
\item[2)] Parks
\item[3)] Gun Range
\item[4)] Fire Department
\item[5)] Police 
\item[6)] Roads
\end{itemize}
\vspace{0.125in}}

\only<4>{\begin{itemize}
\item[1)] \$50 Thousand
\item[2)] \$20 Million 
\end{itemize}
\vspace{1.05in}}

\only<3>{\begin{itemize}
\item[1)] Will request
\item[2)] Requested
\item[3)] Secured
\end{itemize}
\vspace{0.825in}

}

%\only<4>{ \invisible<1-8>{\begin{itemize}
%\item[1)] Will request
%\item[2)] Request
%\item[3)] Secured
%\end{itemize}
%\vspace{0.825in}
%}}


\only<5>{\begin{itemize}
\item[1)] Alone 
\item[2)] w/ Senate Democrat
\item[3)] w/ Senate Republican 
\end{itemize}
\vspace{.825in}
}

\only<6>{\begin{itemize}
\item[1)] Democrat
\item[2)] Republican
\end{itemize}
\vspace{1.05in}
}

\only<7>{\begin{itemize}
\item[] Advertising press release
\end{itemize}
\vspace{1.125in}
}


\pause \pause \pause \pause \pause 

\end{frame}



\begin{frame}
\frametitle{Rewarding Actions and Type of Expenditure, Not Money}

\only<1>{Example  Treatment: \\
\textbf{Headline}: Representative [blackbox] \alert{secured} \alert{\$50 Thousand} \alert{to purchase safety equipment for local firefighters}  \\
\textbf{Body}: Representative [blackbox] (\alert{Democrat}) and \alert{Senator [blackbox], a Democrat}, \alert{secured} \alert{\$50 Thousand} \alert{to purchase safety equipment for local firefighters}.  \\
Rep. [blackbox] said ``This money \alert{will help} \alert{our brave firefighters stay safe as they protect our businesses and homes}" \\}


\only<2>{Example Treatment:\\
\textbf{Headline}: Representative [blackbox] \alert{will request} \alert{\$20 million} \alert{for medical equipment at the local Planned Parenthood.}  \\
\textbf{Body}: Representative [blackbox] (\alert{Democrat}), \alert{will request} \alert{\$20 million} \alert{for medical equipment at the local Planned Parenthood}.  \\
Rep. [blackbox] said ``This money \alert{would help} \alert{provide state of the art care for women in our community.}" \\}

\only<3->{
\invisible<1-2>{\alert{214} other conditions} \\
\invisible<1-3>{Dependent variable: Approve of representative}\\

\vspace{0.175in}
\large 
\invisible<1-4>{\alert{Goal} $\leadsto$ measure effect of credit claiming content on approval ratings }
\invisible<1-5>{Mechanics$\leadsto$  Mechanical Turk sample (\alert{Findings are replicated in representative samples, using real representatives/senators}  )}
}


\pause \pause \pause \pause \pause %\pause \pause 
\end{frame}


\begin{frame}
\frametitle{Rewarding Actions and Type of Expenditure, Not Money}

\begin{itemize} 
\item[-] Participant  $i$ ($i=1, \hdots, N$), has treatment assignment $\textbf{T}_{i}$   \\
\invisible<1>{\item[-] If $\text{T}_{i} = 0$ for control condition} \\
\invisible<1-2>{\item[-] $\textbf{T}_{i} = (\text{T}_{i, \text{type}},  \text{T}_{i, \text{stage} },\text{T}_{i, \text{money}},  \text{T}_{i, \text{collab.}}, \text{T}_{i, \text{part.}} )$ }

\invisible<1-3>{\item[-] $Y_{i}(\textbf{T}_{i} )$ : participant $i$'s Approval decision under treatment $\textbf{T}_{i} $}  
\invisible<1-4>{\item[-] \alert{Quantities of Interest} } 
\only<1-10>{\invisible<1-5>{\item[-] Effect of particular component of message: }
\begin{itemize}
\invisible<1-6>{\item[-] T$_{\text{stage}} = $ Secured}
\invisible<1-7>{\item[-] T$_{\text{stage}}  = $ Requested }
\invisible<1-8>{\item[-] T$_{\text{stage}} = $ Will Request } 
\invisible<1-9>{\item[-] T$_{j} = k $} 
\end{itemize}
}


\invisible<1-10>{\item[-]  Marginal Average Treatment Effect  (MATE$_{\text{T}_{j} = k } $) }  
\end{itemize}
\only<1-13>{
\begin{eqnarray}
\invisible<1-11>{\text{MATE}_{\text{T}_{j} = k} & = & \int \text{E}[Y(\text{T}_{j} = k , \textbf{T}_{-j} )  - Y(0) ] \text{d}\text{F}_{\textbf{T}_{-j} | \text{T}_{j} = k } \nonumber \\}
\invisible<1-12>{\text{MATE}_{\text{T}_{j} = k} & = & \text{E}[Y(\text{T}_{j} = k ) - Y(0) ] \nonumber }
\end{eqnarray}
\vspace{1.5in}

}
\only<14->{   
\begin{eqnarray}
    \text{MATE}_{\text{T}_{j} = k}  & = & \text{E}[Y(\text{T}_{j} = k )|\text{T}_{j} = k ]  - \text{E}[Y(0)|\text{T} = 0 ] \nonumber \\
\invisible<1-14>{\widehat{\text{MATE}_{\text{T}_{j} = k} } & = &    \frac{\sum_{i=1}^{N} Y_{i} I(\text{T}_{ij} = k)  }{\sum_{i=1}^{N} I(\text{T}_{ij} = k) } - \frac{\sum_{i=1}^{N} Y_{i} I(\text{T}_{i}=0)  }{\sum_{i=1}^{N} I(\text{T}_{i}= 0) } \nonumber                }
\end{eqnarray}                    
\vspace{1.5in}
}




\pause \pause \pause \pause \pause \pause \pause \pause \pause \pause \pause \pause \pause \pause 


\end{frame}



\begin{frame}
\frametitle{Rewarding Actions and Type of Expenditure, Not Money}
%(1) Marginal Average Treatment Effect (MATE$_{T_{j} = k } $): $\text{E}[Y(T_{j} = k ) - Y(0) ]$ \\
%\vspace{0.025in} 
%\pause 
\begin{itemize}
\item[-] Response may be conditional on respondent characteristics $\textbf{x}$ \pause \\
\invisible<1>{\item[-] For example $\textbf{x}  = $ (Conservative, Republican)} \pause \\
\invisible<1-2>{\item[-] Marginal Conditional Average Treatment Effect (MCATE$_{\text{T}_{j} = k,\textbf{x}}$)} \pause 
\end{itemize}
\begin{eqnarray}
\invisible<1-3>{\text{MCATE}_{\text{T}_{j} = k,\textbf{x}} & = & \text{E}[Y(\text{T}_{j} = k) - Y(0) | \textbf{x}] }    \nonumber\\
\invisible<1-4>{\text{MCATE}_{\text{T}_{j} = k, \textbf{x}} & = & \text{E}[Y(\text{T}_{j} = k)|\textbf{x} ] - \text{E}[Y(0) | \textbf{x} ] \nonumber }  \nonumber 
\end{eqnarray}


\end{frame}

\begin{frame}
\frametitle{Rewarding Actions and Type of Expenditure, Not Money}
\begin{eqnarray}
\text{MCATE}_{\text{T}_{j} = k, \textbf{x}} & = & \text{E}[Y(\text{T}_{j} = k)|\textbf{x} ] - \text{E}[Y(0) | \textbf{x} ] \nonumber \\
\only<1-8>{\invisible<1>{\widehat{\text{MCATE}_{\text{T}_{j} = k,\textbf{x}}}& = &\frac{\sum_{i=1}^{N} Y_{i} I(\text{T}_{j} = k ,  \textbf{x}_{i} = \textbf{x} )  }{\sum_{i=1}^{N} I(\text{T}_{j} = k , \textbf{x}_{i} = \textbf{x} ) }    - \frac{\sum_{i=1}^{N} Y_{i} I(\text{T}_{i}=0 , \textbf{x}_{i} = \textbf{x})  }{\sum_{i=1}^{N} I(\text{T}_{i}= 0 , \textbf{x}_{i} = \textbf{x}) }  \nonumber } \pause }
\only<9->{
\widehat{\text{MCATE}_{\text{T}_{j} = k,\textbf{x}}} & = & \widehat{g}_{m} (\text{T}_{j} = k, \textbf{x} ) - \widehat{g}_{m}(0, \textbf{x} ) \nonumber 
}
\end{eqnarray}



\begin{itemize}
\invisible<1-2>{\item[-] \alert{Curse of Dimensionality}: highly variable estimates, (sometimes) empty strata} 
\invisible<1-3>{\item[-] Separate systematic differences from noise $\leadsto$ \alert{data} and \alert{assumptions} } \invisible<1-4>{ Heterogeneous treatment effect methods} % ($g_{m} (\text{T}_{j} = k, \textbf{x} ) $)}
\begin{itemize}
\invisible<1-5>{\item[-] LASSO, Find It (Imai and Ratkovic, 2013)$\leadsto$ sparsity} 
\invisible<1-6>{\item[-] Ridge, KRLS (Hainmueller and Hazlett, 2013)$\leadsto$ flexible surface, dense} 
\invisible<1-7>{\item[-] Model $m$ to estimate some function $g_{m}(\text{T}_{j} = k, \boldsymbol{x})$}
\end{itemize}
\invisible<1-8>{\item[-] Perform well: $g_{m}(\text{T}_{j} = k, \boldsymbol{x})$ accurately estimates response surface ($\text{E}[Y(\text{T}_{j} = k)|\textbf{x} ]$)}
\invisible<1-9>{\item[-] Perform well: accurate out of sample prediction and classification (van der Laan et al 2007, Raftery et al 2005) }
\end{itemize}

\invisible<1-10>{Create ensemble: weighting methods by (unique) out of sample predictive performance}

\pause \pause \pause \pause \pause \pause \pause \pause \pause \pause 



\end{frame}

\begin{frame}
\frametitle{Weighted Ensemble to Measure Credit Claiming Rate}

\begin{itemize}
\item[-] Suppose we have $M$ $(m = 1, \hdots, M)$ models.  
\end{itemize}


\begin{eqnarray}
\only<1-13>{\invisible<1>{\widehat{\text{MCATE}_{\text{T}_{j} = k,\textbf{x}}}  & = & \sum_{m=1}^{M} \alert<3>{\widehat{\pi}_{m}} (\alert<4>{\widehat{g}_{m} (\text{T}_{j} = k, \textbf{x} )  }-\alert<4>{\widehat{ g}_{m} (0, \textbf{x} )  })} \nonumber }
\only<14->{\alert{\widehat{\text{MCATE}_{\text{T}_{j} = k,\textbf{x}_{\text{new}}}}  }& = & \sum_{m=1}^{M} \alert{\widehat{\pi}_{m}} (\alert{\widehat{g}_{m} (\text{T}_{j} = k, \textbf{x}_{\text{new}} ) } -\alert{\widehat{ g}_{m} (0, \textbf{x}_{\text{new}} )  } )\nonumber }
\end{eqnarray}

\begin{itemize}
\invisible<1-4>{\item[-] Estimate weights $(\widehat{\pi}_{m})$} 
\only<1-12>{
\begin{itemize}
\invisible<1-5>{\item[-] 10-fold cross validation: generate $M$ out of sample predictions for each observation}
\invisible<1-6>{\item[] $\widehat{\textbf{Y}}_{i}  = (\widehat{Y}_{i1}, \widehat{Y}_{i2}, \hdots, \widehat{Y}_{iM} )$} 
\only<1-11>{\invisible<1-7>{\item[-] Estimate weights with constrained regression:}
\begin{eqnarray}
\invisible<1-8>{Y_{i} & = & \sum_{m=1}^{M} \pi_{m} \hat{Y}_{im} + \epsilon_{i} \nonumber }
\end{eqnarray}
\invisible<1-9>{\item[] where we impose constraints: $\pi_{m} \geq 0 $ and $\sum_{m=1}^{M} \pi_{m} = 1$. } 
\invisible<1-10>{\item[-] Result $\widehat{\pi}_{m}$ for each method } }
\only<12>{\item[-] (Alternatively) Estimate weights from mixture model (EBMA) (Raftery et al 2005;  Montgomery, Hollenback, Ward 2012)$\leadsto$ EM, Gibbs, Variational Approximation} 
\end{itemize}
}
\invisible<1-12>{\item[-] Estimate  $\widehat{g}_{m}(\text{T}_{j} = k, \textbf{x} ) \leadsto $ Apply all $M$ models to entire data set} 

\invisible<1-13>{\item[-] Generate effects  of interest (perhaps weighting to other population)  $\textbf{x}_{\text{new}}$ } 
\end{itemize}
%\invisible<1-14>{Applied to this experiment: Positive weight on three methods:} \\
%\invisible<1-15>{(1) LASSO (0.62), (2) KRLS (0.24), (3) Find It (0.14)} 
%\begin{eqnarray}
%\invisible<1-11>{\text{Pr}(Y_{i} = \text{Credit} | \textbf{w}_{i} ) & = & \sum_{m=1}^{M} \widehat{\pi}_{m} \widehat{g}_{m}(\textbf{w}_{i} ) \nonumber } 
%\end{eqnarray}
%\invisible<1-14>{(Classify if above threshold)} \\
%\invisible<1-15>{{\large \alert{90\% accuracy}} }\\

\pause \pause \pause \pause \pause \pause \pause \pause \pause \pause \pause \pause \pause 
\end{frame}




% \begin{frame}
% \frametitle{Monte Carlo Evidence}
% \begin{center}
% \only<1>{\scalebox{0.4}{\includegraphics{Plot1.pdf}}}
% \only<2>{\scalebox{0.4}{\includegraphics{Plot2.pdf}}}
% \only<3->{\scalebox{0.3}{\includegraphics{Plot2.pdf}}}
% \end{center}
% \only<3->{\large Ensembles outperform constituent methods $\leadsto$ ensembles place weight on  better performing method}

% \end{frame}


\begin{frame}
\frametitle{Returning to \alert{Example} Experiment}

\alert{Recall}: experiment to assess effects of credit claiming on approval $\leadsto$ 1,074 participants (MTurk) \pause  \\
\invisible<1>{Apply ensemble method (7 constituent methods, 10 fold cross validation), including treatments and Partisanship and Ideology.  } \pause \\
\invisible<1-2>{Positive  weight on three methods:} \pause 
\begin{itemize}
\invisible<1-3>{\item[1)] LASSO (0.62)} \pause 
\invisible<1-4>{\item[2)] KRLS (0.24)} \pause 
\invisible<1-5>{\item[3)] Find it (0.14) } 
\end{itemize}

\end{frame}

\begin{frame}
\begin{tikzpicture}
\node (dumm) at (-8, 8) [] {} ; 
\only<1-2, 4, 6>{\node (plot1) at (-5, 8) [] {\scalebox{0.25}{\includegraphics{HetExpIdeoFig2.pdf}}}; }
\only<8>{\node (plot_no) at (-3, 8) [] {\scalebox{0.3}{\includegraphics{SmallTypeMoneyIdeology.pdf}}}; }
\only<9->{\node (plot_no) at (-3, 8) [] {\scalebox{0.2}{\includegraphics{SmallTypeMoneyIdeology.pdf}}}; }
\node (d_plan) at (-8, 4.5) [] {} ; 
\node (d_plan3) at (-1.25, 4.5) [] {} ; 
\node (d_plan2) at (-8, 5.75) [] {} ; 
\node (d_plan4) at (-1.25, 5.75) [] {} ; 

\only<2>{\draw[-, line width = 2pt, red] (d_plan) to [out = 0, in = 180] (d_plan3) ; 
\draw[-, line width = 2pt, red] (d_plan2) to [out = 0, in = 180] (d_plan4) ; }

\only<3>{\node (plot2) at (-5, 8) [] {\scalebox{0.4}{\includegraphics{HetPlanned.pdf}}}; }


\node (d_gun) at (-8, 6.85) [] {} ; 
\node (d_gun3) at (-1.25, 6.85) [] {} ; 
\node (d_gun2) at (-8, 8.05) [] {} ; 
\node (d_gun4) at (-1.25, 8.05) [] {} ; 

\only<4>{\draw[-, line width = 2pt, red] (d_gun) to [out = 0, in = 180] (d_gun3) ; 
\draw[-, line width = 2pt, red] (d_gun2) to [out = 0, in = 180] (d_gun4) ; }

\node (d_gun) at (-8, 9.25) [] {} ; 
\node (d_gun3) at (-1.25, 9.25) [] {} ; 
\node (d_gun2) at (-8, 10.4) [] {} ; 
\node (d_gun4) at (-1.25, 10.4) [] {} ; 


\only<5>{\node (plot3) at (-5, 8) [] {\scalebox{0.4}{\includegraphics{HetGun.pdf}}}; }

\only<6>{\draw[-, line width = 2pt, red] (d_gun) to [out = 0, in = 180] (d_gun3) ; 
\draw[-, line width = 2pt, red] (d_gun2) to [out = 0, in = 180] (d_gun4) ; }

\only<7>{\node (plot3) at (-5, 8) [] {\scalebox{0.4}{\includegraphics{HetPolice.pdf}}}; }



\end{tikzpicture}

\only<9>{$\leadsto$ Constituents evaluate expenditures using \alert{qualitative} information, rather than numerical facts}

\pause \pause \pause \pause \pause \pause \pause \pause 

\end{frame}



\begin{frame}
\frametitle{Estimating Heterogeneous Treatment Effects and the Effects of Heterogeneous Treatments}

Issues with experimental design \pause 
\begin{itemize}
\invisible<1>{\item[-] Treatments are conditional on what else is included: averaging other quantities $\neq$ to excluding other quantities} \pause 
\invisible<1-2>{\item[-] Potential for \alert{fishing} is massive$\leadsto$ \alert{pre analysis plan} and \alert{split-sample design}} \pause 
\invisible<1-3>{\item[-] Assumption about the way information delivered:} \pause 
\begin{itemize}
\invisible<1-4>{\item[1)] We know salient dimensions} \pause 
\invisible<1-5>{\item[2)] We're constructing effects that correspond with effects in reality} 
\end{itemize}
\end{itemize}

\pause 




\end{frame}


\begin{frame}
\frametitle{The Causal Inference Problem}
\begin{itemize}
\item<1-> Two roles: text as \alert{outcome} and text as \alert{treatment}  
\item<4-> Any text analysis requires \alert{mapping} from high to low dimensions 
%\begin{itemize}
%\item we call this a $g$ function
%\item and it is central to the analysis
%\end{itemize} \pause
\item<5-> Need to look at the data to create a good maps 
%\begin{itemize}
%\item<6-> $\rightarrow$ potential for problems like \alert{false discovery}
%\end{itemize} 
\item<6-> Subtle problem: \alert{Fundamental Problem of Causal Inference with Latent Variables (PCILV)} 
\begin{itemize}
\item<7-> $\rightarrow$ use same observations to discover pattern and infer causal effects 
\item<8-> $\rightarrow$ causes key properties of estimators (bias, consistency) to be \alert{undefined}
\end{itemize}
\item<9-> Text as treatment: \alert{always} requires an \alert{exclusion} restriction
\end{itemize}

\begin{center}
\includegraphics<1>[width=.8\textwidth, page=1]{illustrations/TextOutcomeTreatment.pdf} %will be T -> Y
\includegraphics<2>[width=.8\textwidth, page=2]{illustrations/TextOutcomeTreatment.pdf} % will be T -> Documents
\includegraphics<3>[width=.8\textwidth, page=3]{illustrations/TextOutcomeTreatment.pdf} % Documents -> Y
\includegraphics<4-5>[width=.8\textwidth]{illustrations/Gfn.pdf}
\includegraphics<6-9>[width=.8\textwidth]{illustrations/SUTVA.pdf} 
\end{center}

\end{frame}

%\begin{frame}
%\frametitle{The Causal Inference Problem}
%\pause
%\begin{itemize}
%\item Two roles: text as \alert{outcome} and text as \alert{treatment}  \pause
%\item Any text analysis requires \alert{mapping} from high to low dimensions \pause \\
%\begin{itemize}
%\item we call this a $g$ function
%\item and it is central to the analysis
%\end{itemize} \pause
%\item Need to look at the data to create a good maps \pause
%\begin{itemize}
%\item $\rightarrow$ potential for problems like \alert{false discovery}
%\end{itemize} \pause
%\item One subtle problem: \alert{Analyst-Induced SUTVA violation} \pause
%\begin{itemize}
%\item if we build our mapping by looking at the text \pause
%\item $\rightarrow$ mapped outcome depends on \alert{every} unit's treatment assignment \pause
%\item $\rightarrow$ SUTVA violation induced by the \alert{analyst} \pause
%\end{itemize}
%\end{itemize}
%\end{frame}

\begin{comment}
\begin{frame}
\frametitle{An Exciting New Journey of Discovery!}
\begin{center}
\includegraphics[height=.9\textheight]{images/HMS-Discovery.jpg}
\end{center}
\end{frame}

\begin{frame}
\frametitle{With Opportunities for Fishing}
\includegraphics[width=\textwidth]{images/fishing2.jpg}
\end{frame}
\end{comment}

\begin{frame}
\frametitle{What's a social scientist to do?}
\visible<2->{Two Solutions:}
\begin{itemize}
\item<3->[A)] Pre-Analysis Plan: 
\begin{itemize}
\item<4-> Specify mapping ``$g$'' function \textit{a priori} based on some background knowledge. 
\end{itemize}
\end{itemize}
\begin{columns}[c]
\begin{column}{.5\textwidth}
\begin{itemize}
\item<5->[B)] Train-Test Split (Our Plan):
\begin{itemize}
\item<6->[1)] Explicitly set aside a training set for \alert{discovery}. 
\item<8->[2)] Use this \alert{training} set to \alert{develop a $g$ function} that maps high-dimensional data to measurements. 
\item<9->[3)] Given $g$, \alert{estimate} the causal effect in \alert{test} set 
\end{itemize}
\end{itemize}
\end{column}
\begin{column}{.5\textwidth}
\includegraphics<6>[width=.95\textwidth, page=1]{illustrations/TrainingTest.pdf}
\includegraphics<7>[width=.95\textwidth, page=2]{illustrations/TrainingTest.pdf}
\includegraphics<8>[width=.95\textwidth]{illustrations/DevelopingGFn.pdf}
\includegraphics<9>[width=.95\textwidth]{illustrations/Estimation.pdf} % will be estimation
\includegraphics<10>[width=.95\textwidth]{images/HMS-Discovery.jpg}

\end{column}
\end{columns}
\begin{center}
\visible<10>{Train-Test allows for \alert{discovery} while avoiding possibilities of overfitting and PCILV}
\end{center}
\end{frame}

\begin{frame}
\frametitle{Two Running Examples: Treatment and Outcome}

\pause
\begin{columns}[c]
\begin{column}{.5\textwidth}
\begin{framed}
Text as \alert{Treatment} \\
\begin{center}
\includegraphics[height=.3\textheight]{images/Trump.jpg}
\end{center}

What are the features of Trump messages that affect constituents? (Fong and Grimmer 2019)
\end{framed}
\end{column} \pause
\begin{column}{.5\textwidth}
\begin{framed}
Text as \alert{Outcome} \\
\begin{center}
\includegraphics[height=.3\textheight]{images/TrumpBorder.jpeg}
\end{center}

How do presidents going public affect news coverage? (Franco, Grimmer, and Lim 2019)
\end{framed}
\end{column}
\end{columns}
\end{frame}



\begin{frame}
\huge
What features of Trump's rhetoric cause a reaction? 
\end{frame}



\begin{frame}

\scalebox{0.4}{\includegraphics{TrumpScreenTweet.png}}


\end{frame}


\begin{frame}

  \alert{Tweet 1: }

  {\tt
  Why would Kim Jong-un insult me by calling me ``old," when I would NEVER call him ``short and fat?" Oh well, I try so hard to be his friend--and maybe someday that will happen!
  }
  \vspace{0.25in}

  \alert{Tweet 2: }

  {\tt
  Steve Bannon will be a tough and smart new voice at @BreitbartNews...maybe even better than ever before. Fake News needs the competition!
  } \\

  \vspace{0.25in}

\pause
  \Large
  \invisible<1>{Observe difference in evaluations of biographies } \pause \invisible<1-2>{$\leadsto$ Difficult to generalize underlying features (treatments) that drive response}

\end{frame}

\begin{frame}


\alert{Tweet 1: }

  {\tt
  Steve Bannon will be a tough and smart new voice at @BreitbartNews...maybe even better than ever before. \alert{Fake} News needs the competition!
  }

\vspace{0.25in}

\alert{Tweet 1$^{'}$:}

{\tt
  Steve Bannon will be a tough and smart new voice at @BreitbartNews...maybe even better than ever before. \sout{\alert{Fake}} News needs the competition!
  }
\pause 

\vspace{0.125in}

\invisible<1>{\Large Randomly assign 1, 1$^{'}$ and assess response $\leadsto$ are we interested in effect of one word?}


\end{frame}



\begin{frame}

  \alert{Tweet 1: }

  {\tt
Negotiations on DACA have begun. Republicans want to make a deal and Democrats say they want to make a deal. Wouldn't it be great if we could finally, after so many years, solve the DACA puzzle. \alert{This will be our last chance, there will never be another opportunity!} March 5th.  }

  \vspace{0.25in}

  \alert{Tweet 2:}

  {\tt
Negotiations on DACA have begun. Republicans want to make a deal and Democrats say they want to make a deal. Wouldn't it be great if we could finally, after so many years, solve the DACA puzzle. \alert{I will use my office to negotiate a fair deal for the dreamers!} March 5th. } \\
  \vspace{0.25in}

  \Large
  \pause
\invisible<1>{\Large \alert{Latent Representation} (Codebook) $\leadsto$ true whether hand coded, supervised, or unsupervised}

\end{frame}



\begin{frame}
  \frametitle{Text-Based Intervention}

\Large

  \begin{itemize}
    \item[-] Assume ``Interesting'' treatments (coding) must be known in advance \pause   \medskip
    \invisible<1>{\item[-] Discovery of treatments may (often/usually) happen after viewing data} \pause
    \invisible<1-2>{\item[-] \alert{Explicit} discovery phase in research}
  \end{itemize}
 \end{frame}






\begin{frame}
  \frametitle{Three Key Steps}

\pause
  \Large
  \begin{enumerate}
    \invisible<1>{\item[1)] Theory: conditions to identify marginal effects of latent treatments (\alert{Average Marginal Component Effect} (AMCE) is identified) \medskip} \pause
    \invisible<1-2>{\item[2)] Method for discovering features (treatments) \medskip} \pause
    \invisible<1-3>{\item[3)] Method for estimating marginal effect for discovered features (treatments)}
  \end{enumerate}
\end{frame}


\begin{frame}
  \frametitle{Identifying the Marginal Effects of Latent Treatments}

  \Large
  \begin{itemize}
    \item[-] \alert{Average Marginal Component Effect} (AMCE): Isolate effect of one treatment, averaging over other treatments \medskip \pause
    \invisible<1>{\item[-] Let $\boldsymbol{Z}_i$ be $i$'s binary feature vector \medskip}
    \invisible<1>{\item[-] Ex: $\boldsymbol{Z}_i = (0, 0, 1, 1, 0)$} \pause
  \end{itemize}

  \invisible<1-2>{\begin{small}
  \begin{eqnarray}
    \text{AMCE}_{k} & = & \int_{\boldsymbol{Z}_{-k}} \alert<7>{\mathbb{E}}\left[\alert<4>{Y(Z_{k} = 1, \boldsymbol{Z}_{ -k} )}  - \alert<5>{Y(Z_{k} = 0, \boldsymbol{Z}_{ -k} )}\right] \alert<6>{m(\boldsymbol{Z}_{-k})} d\boldsymbol{Z}_{-k} \nonumber
  \end{eqnarray}
  \end{small}}


\invisible<1-6>{\Large Conjoint With Discovered Treatments }\invisible<1-7>{\Large   (or) Discover Features that Drive Response in A/B Test}  
\pause \pause \pause \pause\pause 

\end{frame}




\begin{frame}
  \frametitle{Identifying the AMCE}
  \begin{itemize}
    \item[-] An individual sees a text ($\boldsymbol{X}_i$: text seen by $i$) \medskip
    \item[-] \alert{Function} (assume known for now): text $\leadsto$ treatments in text ($\boldsymbol{Z}_i \equiv g(\boldsymbol{X}_i)$) \medskip
     \item[] $\boldsymbol{Z}_i$ is a low-dimensional rep of $\boldsymbol{X}_i$, describing treatments
  \end{itemize}

\pause
\invisible<1>{
  Assume: \medskip} \pause
  \begin{itemize}
    \invisible<1-2>{\item[1)] No ``spillover" (SUTVA, Rubin 1986: $\tilde{Y}_i(\boldsymbol{X}) = \tilde{Y}_i(\boldsymbol{X}_i)$ )\medskip} \pause
    \invisible<1-3>{\item[2)] Random assignment of texts ($\tilde{Y}_{i}(\boldsymbol{X}_{i}) \independent \boldsymbol{X}_{i}$ for all $i$ )\medskip} \pause
    \invisible<1-4>{\item[3)] Exclusion Restriction: Treatment of interest $g$ is independent of other text features (Grimmer and Fong 2019) $\leadsto$ Vignette experiments} \pause
    \invisible<1-5>{\item[4)] Common support: all combinations of treatments have non-zero probability ($f(\boldsymbol{Z}_i) > 0 $ for all $\boldsymbol{Z}_i \in \text{Range}\; g(\cdot)$)} \pause
  \end{itemize}
  \medskip

  \begin{prop} \label{p:ident}
  \invisible<1-6>{Assumptions 1-4 are sufficient to identify the AMCE$_{k}$ for arbitrary $k$. }
  \end{prop}

\end{frame}



\begin{frame}

\huge

Discovering Treatments and Estimating Marginal Effects

\end{frame}



\begin{frame}
  \frametitle{Discovery of Treatments from Text Corpora}


  \pause
  \begin{itemize}
    \invisible<1>{\item[1)] (Assume) Randomly assign texts, $\boldsymbol{X}_i$, to respondents} \pause
    \invisible<1-2>{\item[2)] Obtain responses $\boldsymbol{Y}_i$ for each respondent}\pause
    \invisible<1-3>{\item[3)] (Randomly) divide texts and responses into training and test set} \pause
    \begin{itemize}
    \invisible<1-4>{\item[a)] Avoid identification issues} \pause
    \invisible<1-5>{\item[b)] Help avoid overfitting }
    \end{itemize}
  \end{itemize}

\end{frame}


\begin{frame}
\frametitle{What Can Go Wrong with the Fundamental Problem of Causal Inference with Latent Variables?}

\pause 

\invisible<1>{\begin{center}
\begin{tabular}{r|cc}
& Treated & Control \\
\hline
Person 1 &  Candidate Morals &  Taxes \\
Person 2 &  Candidate Morals &   Taxes \\
Person 3 &  Polarization &  Immigration \\
Person 4 &  Polarization &  Immigration
\end{tabular}
\end{center}} \pause 

\begin{itemize}
  \item  \invisible<1-2>{$\boldsymbol{T}_{1} = (1, 1, 0, 0)$ $\leadsto$ Categories: Candidate Morals, Immigration} \pause 
  \item \invisible<1-3>{$\boldsymbol{T}_{2} = (1,0,1, 0)$ $\leadsto$ Categories: Candidate Morals, Taxes, Polarization, Immigration } \pause 
\end{itemize}

\invisible<1-4>{Cannot compare categories from $\boldsymbol{T}_{1}$ and $\boldsymbol{T}_{2} \leadsto$ properties of estimator (bias, consistency) not defined! }

\end{frame}


\begin{frame}

\huge 

Discovery method for a $g$


\end{frame}




% \begin{frame}
%   \frametitle{Discovering Interesting Treatments}

% \Large
%   Discovering function from texts to treatments $g()$ \pause
%   \begin{itemize}
%     \invisible<1>{\item[-]  \large Use both documents and responses to discover the function \medskip} \pause
%     \invisible<1-2>{\item[-] \large \alert{Topic} and Supervised \alert{Topic} models workhorse text models (Blei, Ng, and Jordan 2003; Blei and McAuliffe, 2007) \pause }
%   \end{itemize}
%     \invisible<1-3>{ \alert{Treatments on simplex imply marginalization impossible} $\leadsto$ increase in one category implies decrease in other category}



% \end{frame}

% \begin{frame}
%   \frametitle{A Refreshser on sLDA}

%   Each document has a vector of topic proportions.  Proportions add up to 1.

%   \begin{center}
%   \begin{tabular}{| c | c | c |}
%   \hline
%   Education & Family & Occupation\\
%   \hline
%   degree & mother & doctor\\
%   graduated & italian & practice\\
%   science & ancestry & law\\
%   law & n\'ee & business\\
%   economics &  german & work\\
%   \hline
%   \end{tabular}
%   \end{center}

%   Substantive question: Is discussing education beneficial?
% \end{frame}

% \begin{frame}
%   \frametitle{Marginalizing on the Simplex}
%   \begin{itemize}
%     \item Ideal experiment: Compare how much a person likes a biography that is 50\% about education to one that is 30\% about education, all else equal \medskip
%     \item Problem: All else \textit{cannot} be equal \medskip
%     \item Less about education $\implies$ more about family or occupation
%   \end{itemize}
% \end{frame}

\tikzset{
    invisible/.style={opacity=0},
    visible on/.style={alt=#1{}{invisible}},
    alt/.code args={<#1>#2#3}{%
      \alt<#1>{\pgfkeysalso{#2}}{\pgfkeysalso{#3}} % \pgfkeysalso doesn't change the path
  },
}

\begin{frame}
  \frametitle{The Supervised Indian Buffet Process (sIBP, distinct [though related] to Quadrianto et al 2013) }
  \begin{columns}
  \hspace{.3cm}
  \begin{column}{5.25cm}
     \begin{figure}
     \vspace{-1cm}
     \begin{tikzpicture}
     \tikzstyle{main}=[circle, minimum size = 10mm, thick, draw =black!80, node distance = 16mm]
     \tikzstyle{connect}=[-latex, thick]
     \tikzstyle{connect2}=[-latex, thick, red]
     \tikzstyle{box}=[rectangle, draw=black!100]
       \node[main] (Z) [label=right:$Z$] { };
       \node[main] (X) [above right=1.5 and 0.25 of Z,fill=black!25, label=right:$X$] { };
       \node[main] (A) [left=2.3 of X,label=left:$A$] { };
       \node[main] (pi) [left=1.3 of Z,label=left:$\pi$] { };
       \node[main] (Y) [below right=1.5 and 0.25 of Z,fill=black!25, label=right:$Y$] { };
       \node[main] (beta) [above left=.1 and 2.6 of Y, label=left:$\beta$] { };
       \node[main] (tau) [below left=.1 and 2.6 of Y, label=left:$\tau$] { };
       \path (A) edge [connect] (X)
             (Z) edge [connect] (X)
             (Z) edge [connect] (Y)
             (beta) edge [connect] (Y)
             (tau) edge [connect] (Y)
             (pi) edge [connect] (Z)
             (tau) edge [connect] (beta);
       \path[visible on=<2>] (pi) edge [connect2] (Z);
       \path[visible on=<3>] (A) edge [connect2] (X)
                   (Z) edge [connect2] (X);
       \path[visible on=<4>] (beta) edge [connect2] (Y)
                   (tau) edge [connect2] (Y)
                   (Z) edge [connect2] (Y);
     \end{tikzpicture}
     \end{figure}
    \small
    Text and response depend on latent treatments
  \end{column}
  \hspace{.3cm}
  \begin{column}{7.5cm}
    \begin{itemize}
      \small
      \item[-] \alert{Treatment assignment}
      \vspace{-.25cm}
      \small
      \begin{eqnarray}
      z_{i,k} & \sim & \text{Bernoulli}(\pi_{k}) \nonumber\\
      \pi_k & \sim & \prod_{m=1}^{k} \eta_m \nonumber\\
      \eta_m & \sim & \text{Beta}\left(\alpha, 1\right) \nonumber
      \end{eqnarray}
      \item[-] \alert{Document Creation}:
      \vspace{-.2cm}
      \begin{eqnarray}
      \boldsymbol{X}_{i} & \sim & \text{MVN}(\boldsymbol{Z}_{i} \boldsymbol{A}, \sigma^2_{X} I_{D}) \nonumber  \\
      \boldsymbol{A}_k & \sim & \text{MVN}(\boldsymbol{0}, \sigma^2_A I_D) \nonumber
      \end{eqnarray}
      \item[-] \alert{Response}:
      \vspace{-.2cm}
      \begin{eqnarray}
      Y_{i} & \sim & \text{MVN}(Z_{i} \boldsymbol{\beta}, \tau^{-1})\nonumber \\
      \boldsymbol{\beta} | \tau & \sim & \text{MVN}(\boldsymbol{0}, \tau^{-1} I_K) \nonumber\\
      \tau & \sim & \text{Gamma}(a,b) \nonumber
      \end{eqnarray}
    \end{itemize}
  \end{column}
  \end{columns}
\end{frame}

% \begin{frame}
%   \frametitle{Inferring the parameters}
%   \begin{itemize}
%     \item Goal: Obtain $p(\boldsymbol{Z}, \boldsymbol{\pi}, \boldsymbol{A}, \boldsymbol{\beta}, \tau \;|\; \boldsymbol{X}, \boldsymbol{Y}, \alpha, \sigma_X^2, \sigma_A^2, a, b)$ \medskip
%     \item Posterior intractable \medskip
%     \item Use variational approximation \medskip
%     \begin{itemize}
%       \item Use a family of approximating distributions \medskip
%       \item Find member of family closest to true posterior \medskip
%       \item Able to find this without knowing true posterior
%     \end{itemize}
%   \end{itemize}
% \end{frame}

% \begin{frame}
%   \frametitle{Interpreting the Treatments}
%   \begin{itemize}
%     \item Each column of $\boldsymbol{X}$ is standardized to mean 0, sd 1 \medskip
%     \item $X_{i,j} \sim N(\sum_{k=1}^{K} Z_{i,k} A_{k,j}, \sigma_X^2)$ \medskip
%     \item Largest $\boldsymbol{A}_{k,\cdot}$ give words most characteristic of $k$ \medskip
%     \item Ex: If $A_{k,j}$ is large for ``graduated,'' ``college,'' and ``degree,'' $\boldsymbol{Z}_{\cdot, k}$ is education
%   \end{itemize}
% \end{frame}

\begin{frame}
  \frametitle{Discovery of Treatments from Text Corpora}
  \begin{itemize}
    \item[1)] Randomly assign texts, $\boldsymbol{X}_i$, to respondents
    \item[2)] Obtain responses $\boldsymbol{Y}_i$ for each respondent
    \item[3)] Divide texts and responses into training and test set
    \item[4)] In training set: Discover mapping from texts to treatments \pause
    \begin{itemize}
    \invisible<1>{\item[a)] Apply supervised Indian Buffet Process (sIBP) to documents and responses to infer latent treatments in texts  } \pause
    \invisible<1-2>{\item[b)] Model selection via nonparametric process, quantitative fit, and qualitative assessment} \pause
    \end{itemize}
    \invisible<1-3>{\item[5)] In test set: infer treatments and measure their effect} \pause
    \begin{itemize}
    \invisible<1-4>{\item[a)] Use sIBP trained on training set to infer latent treatments on test set documents (without conditioning on test set responses)} \pause
    \invisible<1-5>{\item[b)] Estimate effect of treatments with regression, with a bootstrap procedure to estimate uncertainty}
    \end{itemize}
  \end{itemize}
\end{frame}



\begin{frame}
\frametitle{Trump Tweets}

{\tt YouGov}: survey response to trump tweets \pause 

\only<2>{
  \scalebox{0.4}{\includegraphics{TrumpScreenTweet.png}}
}
\pause 
\begin{itemize}
\invisible<1-2>{\item[-] Survey Equal \# Republicans, Democrats, Independents: read Trump tweet + evaluate ({\tt Great, Good, OK, Bad, Terrible})} \pause 
\invisible<1-3>{\item[-] Aggregate, create scale $[-200,200]$} \pause 
\invisible<1-4>{\item[-] Modify sIBP: Shared treatments, discover heterogeneous effects} \pause 
\invisible<1-5>{\item[-] Train (66\%), Test (33\%), Clustered by tweet} \pause 
\end{itemize} 

\invisible<1-6>{\begin{footnotesize}
\begin{tabular}{|lllll|}
  \hline
 Treatment 1 & Treatment 2 & Treatment 3 & Treatment 4 & Treatment 5 \\ 
  \hline
 fake & cuts & obamacare & flotus & prime \\ 
   news & strange & senators & behalf & minister \\ 
   media & tax & repeal & anthem & korea \\ 
   cnn & luther & healthcare & melania & north \\ 
   election & stock & replace & nfl & stock \\ 
   story & market & republican & flag & market \\ 
   nbc & alabama & vote & prayers & china \\ 
   stories & reform & republicans & bless & executive \\ 
   hillary & record & senate & ready & prayers \\ 
   clinton & high & north & players & order \\ 
   \hline
\end{tabular}
\end{footnotesize}}




\end{frame}


\begin{frame}

\scalebox{0.29}{\includegraphics{images/TrumpTE.pdf}}



\end{frame}


\begin{frame}

\Large 
Sensitivity Analysis: analogous to residual plot in linear regression  \pause \\

\invisible<1>{R Package: {\tt textEffect}} 

\end{frame}



\begin{frame}
\Huge

Text as Outcome


\end{frame}


\begin{frame}

\Huge

How do presidents ``going public" affect public opinion?
\end{frame}


\begin{frame}

\begin{center}
\only<1>{\scalebox{0.45}{\includegraphics{ApprovalPlot.pdf}}}
\only<2>{\scalebox{0.45}{\includegraphics{MIP_Problem.pdf}}}
\only<3>{\scalebox{0.45}{\includegraphics{Topics.pdf}}}
\only<4>{\Huge How do presidents ``going public" affect the media agenda?}


\only<5>{\scalebox{0.6}{\includegraphics{GoingPublicOverall.pdf}}}
\end{center}
\end{frame}


\begin{frame}

\Large

\begin{itemize}
\item[1)] (Assume) random assignment of treatments (use an interrupted time series design) \pause 
\invisible<1>{\item[2)] Obtain text based response $\boldsymbol{Y}_{i}(T_{i})$}
\end{itemize} \pause 

\invisible<1-2>{Function $g$ now uncovers latent features of response: map from text to small number of categories} \pause 

\invisible<1-3>{\begin{eqnarray}
\text{ATE}_{k} & = & \text{E}[ g(\boldsymbol{Y}(1))_{k} -g(\boldsymbol{Y}(0))_{k}] \nonumber 
\end{eqnarray}}


\end{frame}

\begin{frame}
\alert{Discovering (Estimating) Dependent Variable}

\Large
\begin{itemize}
\item[-] Numerous options to discover: hand coding, supervised models, unsupervised models, mixture \pause 
\invisible<1>{\item[-] \alert{All} have same worries: (1) FPCILV (2) Overfitting (potentially via Fishing)} \pause 
\end{itemize}

\invisible<1-2>{\alert{Train/Test Split}} 


\end{frame}


\begin{frame}

\Large
\begin{itemize}
\item[1)] (Assume) random assignment of treatments
\item[2)] Obtain text based response $\boldsymbol{Y}_{i}(T_{i})$ \pause 
\invisible<1>{\item[3)] Randomly split response and text into train/test split} \pause 
\invisible<1-2>{\item[4)] In training set: discover latent dependent variables } \pause 
\begin{itemize}
\invisible<1-3>{\item[a)] Apply Structural Topic Model (Roberts, Stewart, and Airoldi 2017) }\pause 
\invisible<1-4>{\item[b)] Make final model pick based on quantitative model fit and exploration}\pause 
\end{itemize}
\invisible<1-5>{\item[5)] In test set:}\pause 
\begin{itemize}
\invisible<1-6>{\item[a)] Infer dependent variables (using newly available updates to {\tt STM} software (Roberts, Stewart, and Tingley 2017))}\pause 
\invisible<1-7>{\item[b)] Estimate effect of treatments on topic prevalence across categories}
\end{itemize}
\end{itemize}

\end{frame}

\begin{frame}

\Large 
A President's effect on newspaper agenda \pause 
\begin{itemize}
\invisible<1>{\item[-] Response: newspaper articles mentioning {\tt president} in 10 highest circulation papers, two-week window around speech} \pause 
\invisible<1-2>{\item[-] Treatment: Number of days before/after speech article was published} \pause 
\invisible<1-3>{\item[-] 159,217 articles} \pause 
\invisible<1-4>{\item[-] Train: 10\%, Test 90\%} \pause 
\invisible<1-5>{\item[-] Effect estimate: interrupted time series design on topic prevalence (compare share immediately before to share day after)}
\end{itemize}

\end{frame}


\begin{frame}

\begin{center}
\only<2>{\scalebox{0.65}{\includegraphics{AppealEffect.pdf}}}
\only<1>{\scalebox{0.65}{\includegraphics{Announce.pdf}}} 
\end{center}

\end{frame}





\begin{frame}
  \frametitle{Conclusions and Future Directions}

\large
  \begin{itemize}
    \item[-] \Large \emph{Text as Data: How to Make Social Science Inferences Using Language} (Grimmer, Roberts, and Stewart In progress) \pause 
    \begin{itemize}
    \item  \invisible<1>{Sequential (inductive) approach to social science  } \pause 
    \item  \invisible<1-2>{Build and refine theory with \alert{successive} experiments} \pause  
  \end{itemize}
    \item[-] \invisible<1-3> \Large Sensitivity analysis (Fong and Grimmer 2019) \pause 
    \item[-] \invisible<1-4> \Large General Framework: Application to non-text settings (images, voting records, redistricting, and videos) \pause  
    \item[-] \invisible<1-5> \Large Text as Treatment , Text as Outcome , Text as Confounder, Text as Treatment and Outcome
    %\item[-] \Large Division into training and test set is a generally applicable idea even for \alert{causal inference} (see also, Wager and Athey 2015)
  \end{itemize}

\end{frame}



\end{document}
